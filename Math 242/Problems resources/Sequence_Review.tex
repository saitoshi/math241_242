% DOCUMENT FORMATING
\documentclass[12pt]{article}
\usepackage[margin=1in]{geometry}

% PACKAGES
\usepackage{amsmath} % For extended formatting
\usepackage{amssymb} % For math symbols
\usepackage{amsthm} % For proof environment
\usepackage{array} % For tables
\usepackage{enumerate} % For lists
\usepackage{extramarks} % For headers and footers
\usepackage{blindtext}
\usepackage{fancyhdr} % For custom headers
\usepackage{graphicx} % For inserting images
\usepackage{multicol} % For multiple columns
\usepackage{verbatim} % For displaying code
\usepackage{tkz-euclide}
\usepackage{pgfplots}
\usepackage{algorithm}
\usepackage{algorithmic}
\usepackage[ruled,vlined]{algorithm2e}
% SET UP HEADER AND FOOTER
\pagestyle{fancy}
\lhead{\MyCourse} % Top left header
\chead{\MyTopicTitle} % Top center header
\rhead{\MyAssignment} % Top right header
\lfoot{\MyCampus} % Bottom left footer
\cfoot{} % Bottom center footer
\rfoot{\MySemester} % Bottom right footer
\renewcommand\headrulewidth{0.4pt} % Size of the header rule
\renewcommand\footrulewidth{0.4pt} % Size of the footer rule
% ----------
% TITLES AND NAMES 
% ----------

\newcommand{\MyCourse}{MATH 242}
\newcommand{\MyTopicTitle}{Improper Integrals, Sequence, Review}
\newcommand{\MyAssignment}{NAME: \qquad \qquad}
\newcommand{\MySemester}{Spring 2020}
\newcommand{\MyCampus}{University of Hawaii at Manoa}

\begin{document}
\textbf{Directions} Read the directions and show all work. 
\begin{enumerate}
    \item Evaluate the integral or show it diverges
    \begin{itemize}
        \item[(a)] $$\int_{1}^{\infty} \frac{\ln(x)}{x^{4}} dx $$
        \vspace{2cm}
        \item[(b)] $$ \int_{-1}^{1} \frac{dx}{x^{2} -2x} $$
        \vspace{2cm}
    \end{itemize}
    \item List the first five terms of the following sequence
    $$ a_n = \frac{1}{(n+1)!} $$
    \vspace{2cm}
    \item Find a formula for the general $a_n$ of the sequence 
    $\{1,0,-1,0,1,0,-1,0 ... \}$
    \vspace{2cm}
    \item Determine whether the following sequence converge or diverge. If it converges find the limit 
    \begin{itemize}
        \item[(a)] $a_n = 3^n 7^{-n}$ 
        \vspace{2cm}
        \item[(b)] $\{0,1,0,0,1,0,0,0,1\}$
        \clearpage
    \end{itemize}


\subsubsection*{Review before Spring Break}
    \item Given that $\int \frac{1}{1+x^{2}} = \arctan(x) + C$, prove that using trig substitution 
    \vspace{3cm}
    \item \textbf{TRUE or FALSE} Every function has an inverse function. 
    \vspace{1mm}
    \item \textbf{TRUE} or \textbf{FALSE}: When integrating a rational function (partial fraction decomposition), use long division when the degree of the numerator is \textit{greater than} the degree of the denominator.
    \vspace{1mm}
    \item Evaluate the derivative: $xe^{y} = y - 1$
    \vspace{3cm}
    \item Evaluate the integral:
    $$ \int \sin(2z) \cos^{2}(z) dz $$
    \textbf{HINT} Check your trig formula sheet to see what you can substitute with 
    \vspace{3cm}
    \item Tomorrow is Pi Day (3-14). Either list the digit pi (in terms of decimal) to the thousandth place or draw an image of a a pie. 
\end{enumerate}
\end{document}
